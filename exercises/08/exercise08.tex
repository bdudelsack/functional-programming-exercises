\documentclass[a4paper]{article}
\usepackage{../excercises}

\title{\small{8. Übung zur Vorlesung "Fortgeschrittene funktionale Programmierung"}\\\vspace{.5cm}\huge{Monaden}}

\begin{document}

\maketitle

\section{Monad-Instanzen}

\begin{lstlisting}
  class Monad m where
    return :: a -> m a
    (>>=) :: m a -> (a -> m b) -> m b
    (>>) :: m a -> m b -> m b
    m1 >> m2 = m1 >>= \_ -> m2

  instance Monad Maybe where
    return = Just
    Nothing >>= _ = Nothing
    Just x >>= f = f x

  data Identity a = Identity { runIdentity :: a }

  instance Monad Identity where
    return = Identity
    Identity x >>= f = f x

  instance Monad (Either a) where
    return = Right
    Right x >>= f = f x
    Left x >>= _ = Left x
\end{lstlisting}

\section{Listen-Monade}

\begin{lstlisting}
  instance Monad [] where
    return x = [x]
    [] >>= _ = []
    xs >>= f = foldr (\y ys -> f y ++ ys) [] xs

  cross :: [a] -> [b] -> [(a,b)]
  cross as bs = do
    x <- as
    y <- bs
    return (x,y)
\end{lstlisting}

\newpage

\section{State-Monade}

\begin{lstlisting}
  data Tree a = Empty | Node (Tree a) a (Tree a) deriving (Show)

  numberTree' :: Tree a -> Int -> (Tree Int, Int)
  numberTree' Empty v = (Empty, v)
  numberTree' (Node x _ y) v = (Node t1 v2 t2, v2+1)
    where
      (t1, v1) = numberTree' x v
      (t2, v2) = numberTree' y v1

  numberTree :: Tree a -> Tree Int
  numberTree t = t1
    where
      (t1,_) = numberTree' t 1

  data State s a = State { runState :: s -> (a, s) }
\end{lstlisting}

\end{document}
