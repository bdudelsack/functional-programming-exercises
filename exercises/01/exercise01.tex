\documentclass[a4paper]{article}
\usepackage{../excercises}

\title{1. Übung zur Vorlesung "Fortgeschrittene funktionale Programmierung"}

\begin{document}

\maketitle

\setcounter{section}{1}
\section{Größter gemeinsamer Teiler und kleinstes gemeinsames Vielfaches}

\subsection*{Größter gemeinsamer Teiler}

\begin{lstlisting}
ggT :: Integer -> Integer -> Integer
ggT a b
 | b == 0      = a
 | otherwise   = ggT b (a `mod` b)
\end{lstlisting}

\subsection*{Kleinstes gemeinsames Vielfaches}

\begin{lstlisting}
kgV :: Integer -> Integer -> Integer
kgV a b = (a * b) `div` (ggT a b)
\end{lstlisting}

\section{Fibonacci-Zahlen}

\begin{lstlisting}
fib :: Integer -> Integer
fib x
 | x == 0      = 0
 | x <= 2      = 1
 | otherwise   = fib (x-1) + fib (x-2)
\end{lstlisting}

\end{document}